%% LyX 2.3.7 created this file.  For more info, see http://www.lyx.org/.
%% Do not edit unless you really know what you are doing.
\documentclass[12pt,oneside,english]{article}
\usepackage{BP}

\pgfplotsset{compat=1.18}
%\tikzexternalize[prefix=tikz/] 

\title{Homework 3}

\begin{document}

\maketitle

\section*{Assignment 1}
General 2-qubit state can be expressed as:
\begin{equation}
   \ket{\psi} = a \ket{00} + b \ket{01} + c \ket{10} + d \ket{11}.
   \label{stav2qbituniversal}
\end{equation}
If the state is separable following must also hold:
\begin{align}
    \ket{\psi} &= (\alpha \ket{0} + \beta \ket{1})\otimes(\gamma \ket{0} + \delta \ket{1}) \\[.2em]
               &= \alpha \gamma \ket{00} + \alpha \delta \ket{01} + \beta \gamma \ket{10} + \beta \delta \ket{11}
\end{align}

Comparing with \eqref{stav2qbituniversal} leads to system of equation that is solvable iff
\begin{equation}
    ad - cd = 0.
\end{equation}
This is therefore sufficient (and due to binary nature of information about separability also necessary) condition for separability.

\begin{enumerate}[label=(\alph*)]
    \item $ad - bc = 0 - 0 = 0$, state is \textbf{separable}.
    \item $ad-bc = 1-1 = 0$, state is \textbf{separable}.
    \item $ad-bc = -9 + 1 = -8$, state is \textbf{entangled}.
    
\end{enumerate}

\section*{Assignment 2}
Reduced density operators $\rho_A = \t{Tr}_B(\p{\psi})$ and and $\rho_B = \t{Tr}_A(\p{\psi})$ can be easily obtained from projector $\p{\psi}$:
\begin{align*}
    \p{\psi} &= \big[\  9\p{00} + \p{01} + \p{10} + 9\p{11}\\
             &\  + 3\pp{00}{01} - 3\pp{00}{10} - 9 \p{00}{11}\\
             &\  + 3\pp{01}{00} - \pp{01}{10} -3\pp{01}{11}\\
             &\  - 3\pp{10}{00} - \pp{10}{01} + 3\pp{10}{11}\\
             &\  - 9\pp{11}{00} - 3\pp{11}{01} + 3\pp{11}{10} \big].
\end{align*}
Taking the projections needed to get traces one obtains $\rho_A$:
\begin{align}
    \rho_A &= \sum_i \bra{i_B}  \big(\p{\psi}\big) \ket{i_B} \\[.3em]
           &=\tfrac{1}{20} \big[ 10 \p{0} + 10 \p{1} -6\pp{0}{1} -6\pp{1}{0} \big]
\end{align}
and $\rho_B$:
\begin{align} \begin{split}
    \rho_B &= \sum_i \bra{i_B} \big(\p{\psi}\big) \ket{i_B} \\[.3em]
           &=\tfrac{1}{20} \big[ 10 \p{0} + 10 \p{1} +6\pp{0}{1} +6\pp{1}{0} \big].
        \end{split} \end{align} 
Which translates to:
\begin{equation}
    \rho_A = \begin{pmatrix*}[r] \tfrac{1}{2} & -\tfrac{3}{10}  \\[.4em] -\tfrac{3}{10}  & \tfrac{1}{2}  \end{pmatrix*}, \quad
    \rho_B = \begin{pmatrix} \tfrac{1}{2} & \tfrac{3}{10}  \\[.5em] \tfrac{3}{10}  & \tfrac{1}{2}  \end{pmatrix}.
\end{equation}

Theese reduced density matricies can indeed be used to determine whether the state is entangled or not. State is separable if and only if reduced density matricies are density matricies of pure state.
\begin{equation}
    \t{Tr}(\rho^2_A) = \t{Tr}(\rho^2_B) = \tfrac{68}{100} \neq 1
\end{equation}
Trace of the square of the reduced matricies is not equal to 1. Therefore they are not pure density matricies and state $\ket \psi$ is entangled.

Theorem above is easily proved. Consider Schmidt decomposition of state $\ket \psi$:

\begin{equation}
    \ket \psi=\sum_{i}\lambda_{i} \ket i_{A} \ket i_{B}.
\end{equation}
Density matrix then takes form
\begin{equation}
    \rho=\p{\psi} = \sum_{i}\lambda_{i}^{2} \p{i_A} \otimes \p{i_B}.
\end{equation}
Reduced density matrix is then calculated as:
\begin{align} \begin{split}
    \rho_A &= \t{Tr}_B(\rho) = \sum_{i}\lambda_{i}^{2} \,\t{Tr}_B(\ \p{i_A} \otimes \p{i_B} \,) \\[.3em] &= \sum_{i}\lambda_{i}^{2} \, \p{i_A} \t{Tr}_B(\  \p{i_B} \,) = \sum_{i}\lambda_{i}^{2} \, \p{i_A},
\end{split} \end{align}
for $\rho_B$:
\begin{equation}
    \rho_B = \t{Tr}_A(\rho) = \sum_{i}\lambda_{i}^{2} \, \p{i_B}.
\end{equation}
If state $\ket \psi$ is separable, then it has Schmidt number 1. Therefore only one of $\lambda_i$ is not zero. Thus $\rho_A = \p{i_A}$, $\rho_B = \p{i_B}$ and so resuced density matricies represent pure states.

On the other hand, if $\rho_A = \p{i_A}$ and $\rho_B = \p{i_B}$ are pure, it implies $\rho_A = \p{i_A}$, $\rho_B = \p{i_B}$. From equations above it follows that, the schmidt number must be 1 and state is therefore separable.  

\section*{Assignment 3}
Expectation value $\s{ab}$ can be easily obtained by sandwiching operator
\begin{equation}
    \sigma_a \otimes \sigma_b = \sigma_z \otimes (\cos \theta \, \sigma_z + \sin \theta \, \sigma_x) =  \cos \theta \, \sigma_z \otimes \sigma_z + \sin \theta \, \sigma_z \otimes \sigma_x.
\end{equation}
Operator $\sigma_a \otimes \sigma_b$ applied on $\ket \psi$ gives:
\begin{equation}
    \sigma_a \otimes \sigma_b \ket \psi = \tfrac{1}{\sqrt{2}}\big(\sin \theta \,(\ket{00} - \ket{11}) - \cos \theta \, (\ket{01} + \ket{10} ) \big)
\end{equation}
and therefore 
\begin{equation}
    \s{ab} = \swch{\psi}{\sigma_a \otimes \sigma_b }{\psi} = - \tfrac{1}{2} \cos \theta (\bracket{01}{01} + \bracket{10}{10}) = -\cos \theta.
\end{equation}

\section*{Assignment 4}
Proof of the statement is based on the fact that iransformations between orthonormal basis are unitary. Since states $\ket \psi$ and $\ket \varphi$ have same Schmidt numbers, they can differ only in their Schmidt bases. Suppose therefore following expansions:
\begin{equation}
    \ket \psi = \sum_i \lambda_i \ket{i_A} \ket{i_B}, \quad \ket \varphi = \sum_j \lambda_j \ket{j_A} \ket{j_B}.
\end{equation}
States $\ket{i_A}$ and $\ket{j_A}$ (and for B states likewise) form two equivalent bases. Transformations between theese two bases are realised by unitary transformation $U$ (respectively $V$):
\begin{equation}
    \ket{i_A} = U \ket{j_A}, \quad \ket{i_B} = V \ket{j_B}.
\end{equation}
For Schmidt decomposition of $\ket \psi$ it then follows:
\begin{equation}
    \ket \psi = \sum_i \lambda_i \ket{i_A} \ket{i_B} = \sum_j \lambda_j \, U \ket{j_A} \otimes V \ket{j_B} = (U \otimes V) \sum_j \lambda_j \ket{j_A} \ket{j_B} = (U \otimes V) \ket \varphi.
\end{equation}

\section*{Assignment 5}
Entanglement measure is defined as entrophy of reduced density matrix. Applying unitary uperator on Alenas qubit corresponds with applying operator $U \otimes I$ on state $\psi$. Transformation of density operator is therefore:
\begin{equation}
    \rho^\prime = \p{\psi^\prime} = (U \otimes I) \p{\psi} (U \otimes I)^\dag = U_A \, \rho \, U_A^\dag
\end{equation}
as expected since $\rho$ is operator. Transformation of reduced density matricies $\rho_A$ then follows:
\begin{equation}
    \rho_A^\prime = \t{Tr}_B(U_A \, \rho \, U_A^\dag) = U_A \, \t{Tr}_B(\rho) \, U_A^\dag = U_A \, \rho_A \, U_A^\dag,
\end{equation}
and for $\rho_B$ using the fact that partial trace is cyclic in space on which it acts and unitarity property $U_A^\dag U_A = 1$:
\begin{eqnarray}
    \rho_B^\prime = \t{Tr}_A(U_A \, \rho \, U_A^\dag) = \t{Tr}_A(U_A \, \rho \, U_A^\dag) = \t{Tr}_A(\rho \, U_A^\dag U_A) = \t{Tr}_A(\rho) = \rho_B.
\end{eqnarray}
Then for entaglement $E(\ket \psi)$ we get:
\begin{equation}
    E(\ket \psi^\prime) = \t{Tr}_A(\rho_A^\prime \ln{}\rho^\prime_A) = \t{Tr}_A[ U_A \, \rho_A \, U_A^\dag  \ln{} (U_A \, \rho_A \, U_A^\dag) ].
\end{equation}
For function of operator it holds that unitary transformation can be taken out of the argument:
\begin{eqnarray}
    f(A^\prime) = f(U_A \, A \, U_A^\dag) = U_A f(A) U_A^\dag,
\end{eqnarray}
this is direct consequence of either spectral decomposition and the that fact the unitary transformation does not change eigenvalues or of the Taylor expansion and observation that $(U_A \, A \, U_A^\dag)^n = U_A \, A^n \, U_A^\dag$ which can be proven inductively. 
This with cyclic property of the trace implies for the entanglement entrophy $E$:
\begin{align} \begin{split}
    E(\ket \psi^\prime) &= \t{Tr}_A[ U_A \, \rho_A \, U_A^\dag U_A \ln{} (\rho_A)\,U_A^\dag ] = \t{Tr}_A[ U_A \, \rho_A \, \ln{} (\rho_A)U_A^\dag ] \\ 
    &= \t{Tr}_A( \rho_A \, \ln{} (\rho_A)U_A^\dag U_A ) = \t{Tr}_A(\rho_A \, \ln{} \rho_A) = E(\ket \psi)
\end{split} \end{align}
and entanglement therefore remains unaltered.

\end{document}
